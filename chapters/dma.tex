\chapter{DMA}

\paragraph*{}
Mittels Direct Memory Access (DMA) können Daten zwischen Speicherbereichen verschoben werden ohne die CPU zu belasten. Diese Aussage trifft für die DMA des Controllers nicht ganz zu, da die CPU angehalten werden muss um auf den Speicher zuschreiben.

\section*{Aufgabe 26}

\paragraph*{}
Diese Aufgabe stellt eine Lösung der vorhergehende Aufgabe unter der Verwendung der DMA dar. Die DMA-Einheit wird genutzt um Daten zwischen Speicheradressen zu verschieben. Die DMA unterstützt verschiedene MODI. Sie kann Speicherblöcke oder einzelne Zellen übertragen, und diese Übertragung zyklisch wiederholen. Gleichzeitig kann sie auch die Adresse von der sie liest nach jeder Übertragung hoch zählen und verschiedene Interruptquellen nutzen, um die Übertragung zu starten. Meist wird sie verwendet um Ein- oder Ausgabedaten neben läufig von oder zu einem Gerät zu schreiben. In unserem Fall wollen wir das 100 Felder wiederholt lange Array ablaufen und in das Wandlungsregister des DA-Wandlers schreiben. Dazu initialisieren wir alle benötigten Geräte und berechnen wieder alle Ausgabewerte vor. Als Interruptquelle für die DMA verwenden wir den Timer B. Nach dem starten der DMA benötigen wir die CPU nicht mehr, diese könnte nun theoretisch andere Aufgaben bearbeiten.  


\lstinputlisting[caption=aufgabe26.c]{src/aufgabe26.c}

\paragraph*{}
Die Effizienz des Verfahren zeigt sich schon da durch, das wir die Laufzeit des Timers im Vergleich zur vorangegangen Aufgabe erhöhen konnten, obwohl vorher der Code in der ISR aus nur wenigen Zeilen bestand. Für dies Implementierung mit der DMA ist keine ISR mehr nötig.
