\documentclass[11pt,a4paper,onecolumn]{report}
\usepackage{listings}
\usepackage[german]{babel}
\usepackage[utf8]{inputenc}
\title{TI4 - 1. Protokoll}
\author{Rpbin Nehls, Yves M\"uller\\
  Institut f\"ur Informatik,\\
  Freie Universit\"at Berlin\\
  nehls@spline.de uves@spline.de }
\date{}
\begin{document}
\maketitle
\tableofcontents

\part{IO-Ports}

\chapter{Aufgabe 1}

\paragraph*{}
Das Schaltungungsbild zeigt, dass an den Bits Null bis Zwei des Ports vier 
jeweils eine LED so wie der zugehörige Vorwiederstand verschaltet ist. 
Die andere Seite der LEDs ist ist an die Stromquelle (mit vermutlich 3 V) 
angeschlossen. Das vierte Bit des Port vier ist an das Gate eines p-Kanal 
MOSFETs angeschlossen. Solange zwischen Gate und Source keine positive
Spannung anlegt leitet der MOSFET nicht. Die Source des MOSFET ist mit 
der Spannung verbunden. \\
Wird eine Null auf eine Portleitung geschrieben so liegt an ihr, wenn in
PXDIR das entsprechende Bit auf Ausgang steht (der Wert muss Eins sein), 
die Versorgungsspannung des Controllers an, anderenfalls Ground. \\
Wir auf die Bits der LEDs eine Null geschrieben so gibt es ein 
Spannungsgefälle in Leitrichtung der LED so dass ein Strom über die LED 
fließt und diese leuchtet. Anderes herum verhält es sich bei dem 
Lautsprecher; Damit dieser einen Ton ausgeben kann muss über den MOSFET
ein Strom fließen können, dies wird erreicht in dem auf das Gate Spannung 
geleget wird durch schreiben einer Eins in die angeschlossene Protleitung.

 \begin{tabular}{lcr}
  Spalte 1 & Spalte 2 & Spalte 3 \\
  1 & 2 & 3 \\
 \end{tabular}



\begin{tabular}{lr}


test & test  \\

\begin{lstlisting}
#define LEDRT (0x01) 
\end{lstlisting} &
Beschreibung \\

\begin{lstlisting}
P4DIR = 0x00;
\end{lstlisting} &
Beschreibung \\

\begin{lstlisting} 
a = 10;
\end{lstlisting}&
Beschreibung \\

\begin{lstlisting} 
P4OUT = a;
\end{lstlisting}&
Beschreibung \\

\begin{lstlisting}
P4OUT = 0x01;
\end{lstlisting}&
Beschreibung \\



\end{tabular}


\end{document}
